\documentclass[11pt]{article}

\usepackage{amsmath}
\usepackage{amssymb}
\usepackage{verbatim}
\usepackage[margin=0.5in]{geometry}
\usepackage{parskip}

\title{Var gene evolution model}
\author{Ed Baskerville}
\date{26 February 2014}

\begin{document}

\maketitle

\section{Summary}

This document describes the var gene evolution model, and serves as a specification for correct operation of the new C++ implementation.

\section{Overview of model organization}

The model represents a population of hosts. Each host is unique and has a set of current infections by strains and an immune history. Each strain consists of an ordered subset of a global pool of var genes.

\section{Simulation loop}

A simulation follows these steps:
\begin{enumerate}
	\item Load all parameters from parameters file.
	\item Initialize population size with \texttt{initialPopulationSize} hosts. Sample \texttt{nInitialStrains} strains and assign each of them to \texttt{nInitialInfections} hosts.
	\item Set $t = 0$.
	\item Repeat until $t >= $ \texttt{tEnd}
	\begin{enumerate}
		\item Choose and perform the next event via the Gillespie algorithm (or probabilistic equivalent, e.g. next-reaction method), either:
		\begin{enumerate}
			\item Biting events, with rate \texttt{currentBitingRate * currentPopulationSize}, where \texttt{currentBitingRate} is a sinusoidal function of time:
			\begin{enumerate}
				\item Choose a random transmitting host.
				\item Calculate which strains should be transmitted from the transmitting host (see description below).
				\item Modify strains to be transmitted via mutation and/or recombination, possibly with extra circulating strains.
				\item Choose destination host, and transmit strains into host
				\item Within-host dynamics (to be expanded; these may end up taking place on main event queue).
			\item Introduction events, with rate \texttt{currentIntroductionRate * currentPopulationSize}, where \texttt{currentIntroductionRate} is a sinusoidal function of time.
			\item Birth events (if birth-death processes are uncoupled; demography details to be worked out)
			\item Death events (if birth-death processes are uncoupled)
			\end{enumerate}
		\end{enumerate}
	\end{enumerate}
\end{enumerate}

Questions:
\begin{enumerate}
	\item How are host lifetimes chosen---is a gamma distribution reasonable? How should it be parameterized to make the most intuitive sense (shape/scale, shape/mean, shape/mode, something else)? ANSWER: various possible distributions.
	\item Should the population size be constant, with birth coupled to death, or should birth be a continuous process, with instantaneous birth rate equal to death rate (or some other model)? ANSWER: population size may change; allow coupled birth/deaths or drift. This presents mathematical problems; need to match birth rate to death rate to keep population size reasonable (either match mean or match distribution?).
\end{enumerate}

\section{Gene attributes}

Individual var genes may be assigned different attributes:

\begin{enumerate}
	\item transmissibility
	\item mean duration of infection
	\item duration of immunity
\end{enumerate}

\section{Transmission process}

\begin{enumerate}
	\item Calculate P(transmission) of each active strain based on currently expressed var. To begin with, the probability of transmission will be inversely proportional to the number of concurrent infections.
	\item Select strains to be transmitted based on their probability of transmission.
	\item Randomly mutate transmitted strains: each gene in each strain has a constant probability of being sampled from the pool.
	\item Reassort var genes between strains:
	\begin{enumerate}
		\item Select pairs of strains to recombine, so that each pair has a constant probability of recombining. For each chosen pair, generate a daughter strain as a random reassortment of parent strains' genes.
		\item If $n$ strains were picked up from source host, transmit $n$ strains randomly sampled from the original strains and all generated daughter strains.
	\end{enumerate}
\end{enumerate}

\section{Within-host dynamics}

(Adapt from paper writeup.)

\section{Code organization}

\subsection{\texttt{SimParameters.h}}

Class defining all model parameters. Instance variable names are used as keys in a parameters file in JSON format to be loaded into the simulation, e.g.,:

\begin{quote}
\begin{verbatim}
{
  "randomSeed" : 0,
  
  "tEnd" : 100,
  
  "initialPopulationSize" : 100,
  
  "genePoolSize" : 1000,
  
  "introductionRate" : 0.01,
  "bitingRate" : 0.01,
  
  "lifetimeMean" : 1000,
  "lifetimeShape" : 1
}
\end{verbatim}
\end{quote}

\subsection{\texttt{main.cpp}}

Main function, which does the following:
\begin{enumerate}
	\item Opens a SQLite database for model output. (Eventually, it will be able to load an existing SQLite database to start a simulation where it left off.)
	\item Loads parameters from the parameters file.
	\item If the random seed is specified as 0, a new one is generated.
	\item A \texttt{Simulation} object is created and told to run.
\end{enumerate}

\subsection{\texttt{Simulation\{.h, .cpp\}}}

Defines main \texttt{Simulation} class. Main simulation loop is in \texttt{Simulation::run()}, which asks a next-reaction-method sampler to repeatedly sample the next event.

The activity of events will be implemented in auxiliary classes \texttt{BitingEvent} and \texttt{IntroductionEvent}.

\subsection{\texttt{Host\{.h, .cpp\}}}

Representation of host state.

\subsection{\texttt{Strain\{.h, .cpp\}}}

Representation of strains.

\subsection{Other files}

All other files are general-purpose tools for parameter loading, event-based sampling, database management, and other utility functions.

\section{List of all model parameters}

TODO

Notes:
\begin{enumerate}
	\item Time unit: days?
\end{enumerate}

\section{References}

Next-reaction method:

Efficient Exact Stochastic Simulation of Chemical Systems with Many Species and Many Channels. Michael A. Gibson* and and Jehoshua Bruck. The Journal of Physical Chemistry A 2000 104 (9), 1876-1889.

http://dx.doi.org/10.1021/jp993732q



\end{document}
