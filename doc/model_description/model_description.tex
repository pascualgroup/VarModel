\documentclass[11pt]{article}

\usepackage{amsmath}
\usepackage{amssymb}
\usepackage{verbatim}
\usepackage[margin=0.5in]{geometry}
\usepackage{parskip}

\title{Var gene evolution model}
\author{Ed Baskerville}
\date{25 February 2014}

\begin{document}

\maketitle

\section{Summary}

This document describes the var gene evolution model, and serves as a specification for correct operation of the new C++ implementation.

\section{Overview of model organization}

The model represents a population of hosts. Each host is unique and has a set of current infections by strains and an immune history. Each strain consists of an (unordered) subset of a global pool of var genes.

\section{Simulation loop}

A simulation follows these steps:
\begin{enumerate}
	\item Initialize population size with \texttt{initialPopulationSize} hosts. \texttt{nInitialInfections} infections are randomly distributed among hosts.
	\item Set $t = 0$.
	\item Repeat until $t >= $ \texttt{tEnd}
	\begin{enumerate}
		\item Choose and perform the next event via the Gillespie algorithm (or probabilistic equivalent, e.g. next-reaction method), either:
		\begin{enumerate}
			\item Biting events, with rate \texttt{bitingRate * currentPopulationSize}:
			\begin{enumerate}
				\item Choose a random transmitting host.
				\item Calculate which strains should be transmitted from the transmitting host (see description below).
				\item Choose destination host.
				\item Transmit to destination host via one of multiple methods (see description below).
			\end{enumerate}
			\item Introduction events, with rate \texttt{introductionRate * currentPopulationSize}: either
			\begin{enumerate}
				\item Infect a random host with a random new strain.
				\item Infect a random host with a recombination of the two most recent strains to have infected any host in the population.
			\end{enumerate}
		\end{enumerate}
	\end{enumerate}
\end{enumerate}

Questions:
\begin{enumerate}
	\item How are host lifetimes chosen---is a gamma distribution reasonable? How should it be parameterized to make the most intuitive sense (shape/scale, shape/mean, shape/mode, something else)?
	\item Should the population size be constant, with birth coupled to death, or should birth be a continuous process, with instantaneous birth rate equal to death rate (or some other model)?
	\item Should there be different parameters controlling the rate of random introductions and recombination-based introductions?
\end{enumerate}

\section{Transmission}

Question:
\begin{enumerate}
	\item How are transmitting strains determined?
	\item What are all the possible transmission mechanisms?
\end{enumerate}


\section{Code organization}

\subsection{\texttt{SimParameters.h}}

Class defining all model parameters. Instance variable names are used as keys in a parameters file in JSON format to be loaded into the simulation, e.g.,:

\begin{quote}
\begin{verbatim}
{
  "randomSeed" : 0,
  
  "tEnd" : 100,
  
  "initialPopulationSize" : 100,
  
  "genePoolSize" : 1000,
  
  "introductionRate" : 0.01,
  "bitingRate" : 0.01,
  
  "lifetimeMean" : 1000,
  "lifetimeShape" : 1
}
\end{verbatim}
\end{quote}

\subsection{\texttt{main.cpp}}

Main function, which does the following:
\begin{enumerate}
	\item Opens a SQLite database for model output. (Eventually, it will be able to load an existing SQLite database to start a simulation where it left off.)
	\item Loads parameters from the parameters file.
	\item If the random seed is specified as 0, a new one is generated.
	\item A \texttt{Simulation} object is created and told to run.
\end{enumerate}

\subsection{\texttt{Simulation\{.h, .cpp\}}}

Defines main \texttt{Simulation} class. Main simulation loop is in \texttt{Simulation::run()}, which asks a next-reaction-method sampler to repeatedly sample the next event.

The activity of events will be implemented in auxiliary classes \texttt{BitingEvent} and \texttt{IntroductionEvent}.

\subsection{\texttt{Host\{.h, .cpp\}}}

Representation of host state.

\subsection{\texttt{Strain\{.h, .cpp\}}}

Representation of strains.

\subsection{Other files}

All other files are general-purpose tools for parameter loading, event-based sampling, database management, and other utility functions.

\section{List of all model parameters}

TODO

\end{document}
